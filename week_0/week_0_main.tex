%! TEX program = xelatex
\documentclass{beamer}
\usetheme{metropolis}%[sectionpage=none]
%\usepackage{booktabs} 
\usepackage{url}
\def\UrlBreaks{\do\/\do-}
\usepackage{amsfonts, amsmath, lmodern}
\usefonttheme{serif}
\usepackage{algorithm}
\usepackage{algorithmic}


%plots
\usepackage{tikz}
\usepackage{pgfplots}
\usepackage{pgfplotstable}
\pgfplotsset{compat=newest}
\usepackage{subcaption}
\usepackage{csvsimple}

\usepackage[backend=biber]{biblatex}
\bibliography{../bibliography/bibliography.bib}
\renewcommand*{\bibfont}{\footnotesize}

%bibliography numbers
%\setbeamertemplate{bibliography item}{\insertbiblabel}

\newcommand{\teamname}{Team Neville Longbottom}




\title{Image Retrieval for Arguments Using Stance-Aware Query Expansion \texorpdfstring{\cite{kiesel:2021e}}{}}

\author{Bernd Heizmann, Christian Ortlepp, Gustav Lahmann, Max Koch\texorpdfstring{\\$\longrightarrow$ aka \teamname}{aka \teamname}}


\institute{Friedrich-Schiller-Universität Jena}

\date{27.10.2022}

\vfuzz=20pt
\begin{document}
	
	
	\maketitle

	\begin{frame}{Gliederung}
		\tableofcontents
	\end{frame}
	
	\begin{section}{Motivation \& Related Work} %auch related work?
		\begin{frame}{Motivation}
			\begin{itemize}
				\item Bilder spielen eine große Rolle darin den eigenen Standpunkt zu vertreten \& zu verdeutlichen
				\item Beispiele \begin{itemize}
					\item Diagramme
					\item Fotos
					\item Memes
				\end{itemize}
			\item zurzeit gibt es Argumentsuche nur für text (z.B. args.me)
			\end{itemize}
		\end{frame}
	\end{section}
	
	\begin{section}{Methodik}
		\begin{frame}{Methodik}
			\begin{itemize}
				\item def: "Given a keyword query suggesting an issue or a claim for a topic, retrieve as two ranked lists those and only those images that can assist someone in (1) supporting and (2) attacking it."
				\item Bewertungskriterien \begin{itemize}
					\item Topic Relevance
					\item Argumentativeness
					\item Stance relevance
				\end{itemize}
				\item TREC-style Evaluation
			\end{itemize}
		\end{frame}
	
	\end{section}

	\begin{section}{Ansatz: Stance-Aware Query Expansion}
	\begin{frame}{Grundidee}
		\begin{itemize}
			\item Nutzeranfrage mit Stichworten anreichern, welche einen Standpunkt (Pro oder Contra) vertreten
			\item Auswahl der Stichworte wird variiert
		\end{itemize}
	\end{frame}

\begin{frame}{Good-Anti}
	\begin{itemize}
		\item Stichworte: "good" (pro), "anti" (con)
	\end{itemize}
\end{frame}
	
	\begin{frame}{Positive-Negative}
	\begin{itemize}
		\item Stichworte aus MPQA subjectivity Lexikon
		%TODO was ist das?
		\item bis zu 5 Anfragen pro Standpunkt
	\end{itemize}
\end{frame}

	\begin{frame}{Pros-Cons}
	\begin{itemize}
		\item nutzt args.me um Argumente für pro/con zu finden
		\item hängt diese dann an die Suchanfrage an
		%TODO diese Wahrscheinlichkeits-Berechnungnen
	\end{itemize}
\end{frame}
\end{section}
	
	\begin{section}{Ergebnisse}
		\begin{frame}{Ergebnisse}
			\begin{itemize}
				\item fast alle (92-95\%) Bilder on-topic
				\item argumentative precision@10: \begin{itemize}
					\item good-anti 0.64
					\item pros-cons 0.52
				\end{itemize}
				\item stance precision@10 16-18\% niedriger
			\end{itemize}
		\end{frame}
	
	\begin{frame}{Probleme}
		\begin{itemize}
			\item Bilder oft eher illustrierend als argumentativ
			\item teilweise zu viele Bilder von Shopping-Angeboten
			\item einige Standpunkte werden online eher einseitig argumentiert
		\end{itemize}
	\end{frame}
	\end{section}
	
	\begin{frame}[allowframebreaks]{Bibliography}
		\printbibliography
	\end{frame}
	%TODO Bilder?
	%TODO Stichpunkte auf Englisch?
	
\end{document}
