%% The first command in your LaTeX source must be the \documentclass command.
%%
%% Options:
%% twocolumn : Two column layout.
%% hf: enable header and footer.
\documentclass[
% twocolumn,
% hf,
]{ceurart}

%%
%% One can fix some overfulls
\sloppy

%%
%% Minted listings support 
%% Need pygment <http://pygments.org/> <http://pypi.python.org/pypi/Pygments>
\usepackage{listings}
%% auto break lines
\lstset{breaklines=true}

%%
%% end of the preamble, start of the body of the document source.
\begin{document}

%%
%% Rights management information.
%% CC-BY is default license.
\copyrightyear{2023}
\copyrightclause{Copyright for this paper by its authors.
  Use permitted under Creative Commons License Attribution 4.0
  International (CC BY 4.0).}

%%
%% This command is for the conference information
\conference{CLEF 2023: Conference and Labs of the Evaluation Forum, 
    September 18--21, 2023, Thessaloniki, Greece}

%%
%% The "title" command
\title{Image Retrieval for Arguments - Team "Neville Longbottom"}
\title[mode=sub]{Notebook for the Touch{\'e} Lab on Argument and Causal Retrieval at CLEF 2023}


%%
%% The "author" command and its associated commands are used to define
%% the authors and their affiliations.
\author[1]{Dascha Elagina}[%
email=daria.elagina@uni-jena.de
]
\author[1]{Bernd-Albrecht Heizmann}[%
email=bernd-albrecht.heizmann@uni-jena.de
]
\author[1]{Max Koch}[%
email=max.koch@uni-jena.de
]
\author[1]{Gustav Lahmann}[%
email=gustav.lahmann@uni-jena.de
]
\author[1]{Christian Ortlepp}[%
email=christian.ortlepp@uni-jena.de
]


\address[1]{Friedrich-Schiller University Jena,
07743, Jena}


%%
%% The abstract is a short summary of the work to be presented in the
%% article.
\begin{abstract}
  A clear and well-documented \LaTeX{} document is presented as an
  article formatted for publication by CEUR-WS in a conference
  proceedings. Based on the ``ceurart'' document class, this article
  presents and explains many of the common variations, as well as many
  of the formatting elements an author may use in the preparation of
  the documentation of their work.
\end{abstract}

%%
%% Keywords. The author(s) should pick words that accurately describe
%% the work being presented. Separate the keywords with commas.
\begin{keywords}
  CLIP \sep
  ChatGPT \sep
  IBM Debater \sep
  args.me
\end{keywords}

%%
%% This command processes the author and affiliation and title
%% information and builds the first part of the formatted document.
\maketitle

\section{Query Expansion}

\subsection{ChatGPT}%Christian
\subsection{ArgsME}%Bernd
\subsection{Pro-Con-Org?}%Dascha
\section{Retrieval}

\subsection{CLIP}%Max

One of the ways we retrieved images of the dataset given topic is by using the CLIP model\cite{DBLP:journals/corr/abs-2103-00020}. We use it to first calculate an image vector containing 512 entries for each image in the dataset. Doing the same for a query (topic), it can output a text-image similarity score by computing the cosine similarity between both vectors. We calculated the image vectors in advance and calculate the query vector only after the topic is given, taking very short time to compute the score. Given normalized vectors, the score can also be calculated by finding the distance between both vectors, assuming they start at the same point in a 512-dimension coordinate system.

Our query is simply the given topic question. By changing it to be more argumentative for a specific stance, we could not achieve improvements in stance presicion.

\subsection{BM25}%Gustav
%content extraction beschreiben
\subsection{BM25 Query Refinement?}

\section{Reranking}

\subsection{IBM Project Debater}

We used IBM's Project Debater Early Access API to score the text content of a Website.

For this we used the "pro-con" endpoint, which given two sentences returns the stance of the first sentence relative to the second sentence as a number between -1 (CON) and 1 (PRO).

We applied this by first splitting the website text content into sentences and then ranking each sentence of the document relative to the current search query. The stance of the document as a whole is then computed as the average of all individual text stances.
%todo reference project debater website

\subsection{Sentiment Analysis}%Max
%todo probieren mit neuer content extraction
\section{Evaluation}

\section{Results}

\end{document}
